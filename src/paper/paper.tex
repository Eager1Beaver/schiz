\documentclass[14pt]{article}
\usepackage[utf8]{inputenc}
\usepackage{listingsutf8}
%\usepackage[T2A]{fontenc}
\usepackage[english]{babel}
\usepackage{amssymb,amsmath,amsfonts,amsbsy}
%\usepackage{mathtext}
%\usepackage{indentfirst}
\usepackage{tabularx}
\usepackage{booktabs}
\usepackage{graphicx}
\usepackage[center]{caption}
\usepackage{subcaption}
\usepackage{cite}
\usepackage{enumerate}
\usepackage{enumitem}
%%\usepackage[braket]{qcircuit}
\usepackage{marvosym}
\usepackage{multicol}
\usepackage{comment}
\usepackage{authblk}
\usepackage{lipsum}
%\usepackage{hyperref}
\graphicspath{{figs/}}
%\DeclareGraphicsExtensions{.pdf,.png,.jpg}
%\bibliographystyle{utf8gost780u}
%\usepackage{apacite}
%\bibliographystyle{apacite}

% Print line number page wisely 
%\usepackage[pagewise]{lineno}
%\linenumbers

%\renewcommand\linenumberfont{\normalfont\bfseries\normal}
%\renewcommand\linenumberfont{\normal}

\title{\textbf{Schiz} Project}

\author[1]{Ilia Golub}

%\author[1]{Pratit Kandel}

\author[1]{Prosperity Oguama}

\affil[1]{Gonda Interdisciplinary Brain Research Center, Bar-Ilan University, Ramat Gan, Israel}
%\affil[4]{Institute of Natural Sciences and Mathematics, Ural Federal University, Ekaterinburg, Russia}
%\affil[5]{Department of Cardiology, Clinical Sciences, Lund University, Lund, Sweden}
%\affil[6]{Arrhythmia Clinic, Skane University Hospital, Lund, Sweden}
%\affil[*]{Correspondence: golub.ilia.ilya@gmail.com}

\setcounter{Maxaffil}{0}
\renewcommand\Affilfont{\itshape\small}
\date{}

\usepackage[onehalfspacing]{setspace}

\usepackage{listings}
\renewcommand{\lstlistingname}{Listing}
\lstset{language=Python,
        numbers=left,
        basicstyle=\small\ttfamily,
        breaklines=true,
        showstringspaces=false,
        stepnumber=1,         
        numbersep=5pt,          
        showspaces=false,       
        showtabs=false,         
        %frame=single,          
        tabsize=4,              
        captionpos=b,           
        breakatwhitespace=false,
        escapeinside={\%*}{*)}, 
        texcl=true
        }

\usepackage[left=2.5cm, right=1cm, top=1cm, bottom=2.5cm]{geometry}

\usepackage[
    colorlinks=true,
    linkcolor=blue,
    filecolor=magenta,      
    urlcolor=cyan,
    pdftitle={CW-2},
    pdfauthor={Golub I},
    bookmarks=true,
    linktoc=all,
	final]{hyperref}

\usepackage[nottoc]{tocbibind}

\makeatletter
\renewcommand{\@biblabel}[1]{#1.}
\makeatother

\sloppy

\righthyphenmin=2

\renewcommand{\theenumi}{\arabic{enumi}.}
\renewcommand{\labelenumi}{\arabic{enumi}.} 
\renewcommand{\theenumii}{\arabic{enumii}.} 
\renewcommand{\labelenumii}{\arabic{enumi}.\arabic{enumii}.}
\renewcommand{\theenumiii}{\arabic{enumiii}.} 
\renewcommand{\labelenumiii}{\arabic{enumi}.\arabic{enumii}.\arabic{enumiii}.} 


\DeclareMathOperator{\rot}{rot}		% Определяем операции rot
\DeclareMathOperator{\diverg}{div}	% div
\DeclareMathOperator{\grad}{grad}	% grad
\DeclareMathOperator{\arth}{Arth}    % arc tanh
\DeclareMathOperator{\Ai}{Ai}
\DeclareMathOperator{\Bi}{Bi}

\setcounter{tocdepth}{1}

\begin{document}

% Title
\maketitle

% Abstract
\section{Abstract}

\textbf{Background}\\
\lipsum[1]
\textbf{Aim}: ...

\textbf{Methods}\\
\lipsum[1]

\textbf{Results}\\
\lipsum[1]

\textbf{Conclusion}\\
\lipsum[1]

\textbf{Keywords: schizophrenia, sMRI, CNN}

% Introduction
\section{Introduction}

Schizophrenia (SZ) is a chronic mental disorder affecting 1 in 300 people worldwide (Harestad, 2024), characterized by disruptions in thought processes, perception, and social functioning. Accurate and timely diagnosis is crucial for effective management, yet schizophrenia remains one of the most challenging disorders to diagnose due to its heterogeneity and overlap with other psychiatric conditions (Joyee et al,; Benlee & Adac). Being linked to structural and functional brain abnormalities, SZ has been actively studied using Magnetic Resonance Imaging (MRI). Structural MRI has contributed to our understanding of the neural basis of SZ, highlighting abnormalities such as ventricular enlargement, gray and white matter alterations, frontal cortex enlargement, and cortical thickness (Joyee et al.; Teskera & Bozek.,2023).

%neuroimaging has emerged as a powerful tool for identifying biomarkers associated with schizophrenia. 

The advent of machine learning (ML) and deep learning (DL) techniques has further accelerated efforts to automate schizophrenia diagnosis by analyzing complex neuroimaging data.

Despite the significant progress in developing convolutional neural network (CNN) architectures for schizophrenia detection [REFS!!!], a critical challenge persists: the lack of standardization in preprocessing and feature extraction pipelines. Variability in preprocessing methods---such as noise removal, intensity normalization, and artifact correction---introduces inconsistencies in model training and evaluation.

Existing literature primarily emphasizes developing DL models, often neglecting to explicitly detail the preprocessing steps applied to the neuroimaging data. This gap is particularly concerning, as preprocessing directly impacts the quality and reliability of input data, influencing downstream analyses and model performance. Studies leveraging benchmark datasets,[REFS!!!] such as BrainGluSchi, COBRE, MCICShare, NMorphCH, and NUSDAST frequently report high classification accuracies, but the preprocessing pipelines remain inadequately documented or inconsistent, further complicating efforts to replicate findings.

%Similarly, differences in feature extraction techniques hinder the interpretability and generalizability of machine learning outputs, as they may target inconsistent brain regions or utilize diverse spatial features. These issues contribute to variability in reported results across studies, limiting reproducibility and comparability.

To address these challenges, this project aims to systematically experiment with various preprocessing techniques commonly used in medical imaging, such as contrast enhancement, noise removal, and data augmentation. By evaluating their impact on schizophrenia detection from MR images, we seek to propose a robust and reproducible preprocessing pipeline. Standardizing preprocessing practices could enhance the reproducibility of ML-based schizophrenia diagnostics, ultimately contributing to more reliable and clinically applicable models.

% Methods
\section{Methods}

\subsection{Dataset and Exploratory Data Analysis}

The raw data were obtained from the \href{http://schizconnect.org}{SchizConnect} database (accessed January 1, 2025), which houses structural and functional MRI data. The database provides filtering options to enable users to select data that meets specific criteria including MRI field strength and clinical diagnosis (e.g. schizophrenia, bipolar disorder).

The data obtained originates from the Center of Biomedical Research Excellence (COBRE) [REF!!!] and MIND Clinical Imaging Consortium (MCICShare) [REF!!!] datasets in NifTI format. It includes data from 310 individuals, 153 with schizophrenia and 157 healthy controls. [CHECK!!!]

The data from SchizConnect was filtered based on MRI field strength and clinical diagnosis. We selected only images captured with 3T MRI scans.

Clinical data was made available in form of csv files. The clinical and demographic features of the dataset used in this project are presented in Table~\ref{tab:cobre_mcicshare_clinical_demographic}.

The demographics of the COBRE and MCICShare datasets, including age and gender distribution are similar.

The COBRE dataset contains images of patients in three diagnosic categories: schizophrenia sctrict, schizoaffective, and no known disorder (healthy controls). MCICShare contains images of patients diagnosed with schizophrenia broad and no known disorder. 
???However, previous works suggest trying to identify the distinct classes by applying machine learning and deep learning models on only MRI data does not yield good results (Harestad, 2024). 
Therefore, we grouped all the schizophrenia classes as one. Figures 4 and 5 show the age distribution before and after combining the schizophrenia classes respectively. Figure 6 shows the final gender distribution of the combined dataset.


% Figs should not duplicate the tables and vice versa

\begin{center}
	\begin{table}
        \centering
        \caption{\label{tab:cobre_mcicshare_clinical_demographic}Description!!!.}
        \begin{tabular*}{500pt}{@{\extracolsep\fill}lcccccc@{\extracolsep\fill}}
            \toprule
            & \multicolumn{2}{c}{COBRE} & \multicolumn{2}{c}{$MCICShare$} & \multicolumn{2}{c}{$Combined$}
            \\\cmidrule{2-4}\cmidrule{5-7}
            & \textbf{Individuals with SZ (n=83)} & \textbf{Healthy controls (n=105)} & \textbf{Individuals with SZ (n=70)} & \textbf{Healthy controls (n=52)} & \textbf{Individuals with SZ (n=153)} & \textbf{Healthy controls (n=157)} \\
            \midrule
            Minimum age             & 19  & 18  & 18	& 18 & 18 & 18 \\
            Maximum age 		    & 66  & 65 & 56 	& 60 & 66 & 65 \\
            Average age             & 38.2 & 38.7 & 35.5	& 36.6 & 37.0 & 38.0	\\
            Gender (Male/ Female)   & 68/15 & 75/30 & 51/19	& 30/22 & 119/34 & 105/52	\\
            \bottomrule
        \end{tabular*}
    \end{table}    
\end{center}

%
\subsection{Image Preprocessing}

% General thoughts
Why Preprocess MRI Scans?
Consistency Across Scans: MRI data often comes from different scanners or sessions, leading to variations in intensity ranges, resolutions, and voxel spacing.
Noise Reduction: Preprocessing can reduce irrelevant noise, helping you calculate more meaningful SNR and PSNR values.
Standardization: Normalized and resampled data ensures fair comparisons between scans.
Artifact Removal: Some preprocessing steps (e.g., brain extraction, cropping) focus metrics on the brain region, ignoring irrelevant background areas.

To perform a grid search to find the best combination of parameters for the resample_image function based on signal-to-noise ratio (SNR), you can use a custom test function.

\subsubsection{Resampling}
Resample to a Standard Resolution

Structural MRI datasets often vary in resolution. Resampling ensures uniform input for the model.

Resample all scans to a common voxel spacing or resolution. This ensures that metrics like SNR and PSNR are calculated on comparable spatial dimensions.

Why? Different resolutions can lead to different noise and signal profiles.

\subsubsection{intensity normalization}

Normalize voxel intensities to a consistent range (e.g., z-score normalization or min-max scaling).
This method scales the data to a range of [0, 1]

Normalize the intensity values of all scans to a consistent range (e.g., [0, 1] or [0, 255]). This ensures that metrics like PSNR are not skewed by varying intensity ranges.

Advantages:
Produces values in a fixed range, which many models expect.
Maintains the relative scale of the data.
Disadvantages:
Sensitive to outliers, as extreme values in the data will dominate the scaling.

You can improve the results of compute_brain_mask by adding preprocessing steps or tweaking its parameters:
Normalize the intensity values of the MRI data before passing it to compute_brain_mask.
This helps the algorithm better distinguish between brain and non-brain regions.

\subsubsection{Brain extraction}

Remove non-brain tissues 

To create a function that automatically determines where the brain part is and crops the 3D MRI volume so that the brain remains intact, you can leverage the brain mask generated during brain extraction. 

When Brain Extraction is Necessary
Focus on Brain Structures:
If your neural network is specifically designed to analyze brain structures (e.g., cortical thickness, subcortical volumes, lesions), extracting the brain helps eliminate irrelevant information, such as the skull or surrounding tissues, which could confuse the model.
Many brain-specific datasets and pretrained models (e.g., for segmentation or classification) expect inputs to have only the brain region.
Reduce Noise:
Removing non-brain regions minimizes noise, making it easier for the network to focus on relevant features.
This step can enhance performance, especially in tasks where small differences in brain anatomy are crucial (e.g., schizophrenia detection).
Standardization:
Brain extraction ensures that the input images are more consistent in terms of content, which can help the model generalize better.

% Discuss When Brain Extraction May Not Be Necessary

The nobrainer tool offers pre-trained deep learning models for brain extraction and is designed for robust performance on diverse datasets.
ANTsPyNet is a Python library that offers neural network architectures and pre-trained models for medical image analysis, including brain extraction.

Apply Morphological Operations:
Use scipy.ndimage to refine the mask by closing gaps or removing small regions.

\subsubsection{Cropping}

Crop or pad images to a fixed size for uniformity across the dataset.
Crop the brain region to a consistent size across scans. This step is particularly useful for metrics that compare local intensities.

\subsubsection{Smoothing}

Smoothing and noise removal can significantly enhance the quality of structural MRI data, improving the signal-to-noise ratio (SNR) and making downstream tasks more robust.
Why Smooth?
Reduce Noise: Structural MRI images often contain random noise that can affect feature extraction and model training.
Enhance Signal: Smoothing can emphasize meaningful structures, such as anatomical boundaries.
Standardize Images: Makes images consistent and reduces variability introduced by acquisition artifacts.

How to Smooth?
Gaussian Smoothing 
Use a Gaussian filter to smooth images while preserving edges to some extent.

Apply a Gaussian filter to reduce high-frequency noise that might artificially inflate the SNR or PSNR.

Median Filtering
Effective for removing salt-and-pepper noise.

Wavelet Denoising
Advanced technique that removes noise while preserving details.

Task-Specific Requirements:
For tasks requiring high anatomical precision (e.g., detecting small abnormalities), keep smoothing minimal.
For classification tasks, moderate smoothing may help by focusing on broader patterns.

\subsubsection{Validation}

validating the results of preprocessing is crucial. Without validation, you risk introducing artifacts or losing critical information, which could undermine the effectiveness of your model. Here’s a structured approach to validate preprocessing:

1. Visual Inspection
Why? To ensure preprocessing steps don't introduce artifacts or distort anatomical structures.
How?
Plot a few slices of the MRI before and after each preprocessing step.

2. Quantitative Metrics
Why? To assess the impact of preprocessing on the data's integrity and quality.
How?
Calculate metrics before and after preprocessing to track changes.

2.1Signal-to-Noise Ratio (SNR):
Evaluate how noise levels are reduced while retaining the signal.
SNR: Measures the ratio of signal to noise in a single image. It does not require a second image for comparison.

SNR Definition:
In the SNR calculation, you're explicitly defining signal (mean intensity in the brain region) and noise (standard deviation in the background or non-brain region).
This method isolates the contributions of signal and noise effectively, and thus SNR tends to be higher when the signal is strong relative to noise.

Refine Noise Estimation
Mask-Based Noise Calculation: Use a binary mask to explicitly select a background region for noise estimation.
Outside-Brain Noise: If you have brain masks, calculate noise statistics from outside the brain.
Consider more robust methods like Otsu's thresholding 

Otsu's Threshold is Too Aggressive:
Otsu's method focuses on separating background from foreground but might miss subtle intensity gradients within the brain.
1. Adjust the Threshold Dynamically
Instead of relying entirely on Otsu's threshold, apply a scaling factor to relax it slightly:

2. Combine Threshold Methods
Use Otsu’s threshold as a base and add a secondary, intensity-based threshold to ensure regions with slightly lower intensities are included

PSNR: Compares two images and is useful for assessing image reconstruction, compression, or denoising performance.

PSNR Definition:
PSNR compares the similarity between two images (e.g., before and after smoothing).
It does not explicitly differentiate between signal and noise like SNR does.
Instead, it uses the difference between the original image (or reference) and the modified image (e.g., smoothed) to calculate the MSE and, ultimately, PSNR.

Alternative: Relative PSNR for One Image
If you have a hypothetical "ideal" signal (e.g., the maximum intensity value as a baseline), you could calculate a relative PSNR for one image. This relative PSNR could be used as a way to assess the "quality" of a single image compared to a theoretical perfect image of uniform intensity.

2. Contrast-to-Noise Ratio (CNR):
Compare intensities between regions (e.g., gray matter vs. white matter). Contrast-to-Noise Ratio (CNR): Measures the contrast between regions of interest (e.g., brain vs. background).

3 Histogram Analysis:
Inspect intensity distributions to ensure normalization has not distorted the data.

In addition to SNR, consider these metrics to evaluate resampled images:
Mean Squared Error (MSE): Compare resampled and original images voxel-wise to evaluate accuracy.

rmse
One common approach is to use an idealized reference image, such as an image where all the pixels have the maximum intensity (or another constant value) for comparison

Structural Similarity Index (SSIM): Quantify structural similarity between the resampled and original images.
when calculating the Structural Similarity Index (SSIM), the goal is to maximize the SSIM value, as a higher SSIM indicates greater similarity between the two images.


1. Metrics for the Entire 3D Volume (Your Current Approach):
When to Use:
If the analysis aims to assess the overall quality or similarity of the entire MRI dataset.
When treating the volume as a whole is more relevant (e.g., for registration, segmentation, or other volumetric tasks).
Advantages:
Captures global differences and variations across the entire dataset.
Simpler and faster to compute since it avoids slice-wise computations.
Disadvantages:
May obscure slice-level differences (e.g., a few noisy slices might not significantly affect the overall metrics but could be important diagnostically).
% Discuss Metrics for Each 2D Slice (Slice-wise Approach):

%
\subsection{Data Augmentation}
Here are some common transformations you can apply to your MRI images, tailored to medical imaging:
a. Flips
Horizontal Flip (RandomHorizontalFlip): Good for brain MRI images, as the brain is symmetric along the sagittal plane.
Vertical Flip (RandomVerticalFlip): Less commonly used, but still possible, depending on the orientation of your images.
b. Rotations
RandomRotation: Useful for handling minor variations in patient positioning during scans.
Angle Range: Typically, 10°–30° is sufficient. Avoid large angles (>45°), as they may produce unrealistic data.

c. Random Affine Transform
Includes translation, scaling, rotation, and shear.
Translation: Small translations (e.g., up to 10% of image size) simulate minor misalignments.
Scaling: Scale variations between 0.9x and 1.1x are typical.
Shear: Small shearing angles (e.g., ±10°) can add variety.
d. Intensity-Based Augmentations
RandomBrightness or RandomGamma: Simulates variations in scan intensity due to different machines or settings.
Gaussian Noise: Adds noise to simulate variations or imperfections in the imaging process.

f. Cropping and Resizing
Center cropping or random cropping to extract relevant parts of the brain.
Resize all images to a fixed size (e.g., (128, 128, 128) or (256, 256, 256)) for consistency.

Brightness and Contrast: Adjusting these properties can simulate variations in scanner settings, but it must be done carefully. Too much adjustment may distort the structural information critical for your task.

For MRI images, adding noise or slight intensity shifts is a more domain-specific way to introduce variation:


2. Choosing Parameters
The parameters depend on your dataset and the expected variations in real-world data:

Domain Knowledge: Consult with radiologists or domain experts to understand typical variations in brain MRI scans.
Visualize Transformed Data: Plot augmented samples to ensure the transformations don't distort anatomical features.
Experimentation: Start with conservative parameters (e.g., small rotations, flips) and gradually expand based on model performance.

4. Validation Images
No Augmentation: Do not apply augmentation to validation or test images. Simply normalize and resize them for consistency:

%
\subsection{CNN architecture}

%
\subsection{CNN validation}

% Results
\section{Results}

\subsection{Image Preprocessing}

An SNR greater than 1 means that the average signal is more intense than the noise.

MRI SNR Values
For MRI, typical SNR values in literature are often much higher than 1:
Clinical MRIs: SNR values between 10–40 for good-quality scans.
Research MRIs: Values can exceed 50 or even 100 in highly optimized settings.
Values below 10 may indicate noisy or poor-quality scans.

PSNR Behavior:
PSNR should be higher when the image quality is better, which typically results in a greater difference between the actual data and the ideal reference.
If you have well-normalized data and a clean signal with little noise, PSNR will naturally be higher because it’s measuring how close the data is to an ideal version of the image.
Expected Value Ranges:
SNR values typically range between 10 and 40 for good quality MRI images.
PSNR values, when properly calculated, can be much higher because they consider an ideal signal. For instance, PSNR values between 20 and 50 dB are common, but they depend heavily on image resolution, noise, and contrast.

How to Interpret These Results:
SNR = 100: Indicates that the image has a strong signal relative to noise. This is expected after smoothing because smoothing reduces noise.
PSNR = 45: Indicates that the smoothed image differs moderately from the original. This suggests the smoothing process significantly altered the pixel intensities.

\subsection{Data Augmentation}

\subsection{CNN performance}

% Random text
\lipsum[5-6]

% Discussion
\section{Discussion}

% Random text
\lipsum[7-8]

% Conclusion
\section{Conclusion}

% Random text
\lipsum[9-10]



% Formalities
\section*{Acknowledgements}

...

\subsection*{Author contributions}

\textbf{Conception}: Prosperity Oguama, Ilia Golub; \textbf{Data Acquisition}: Ilia Golub; \textbf{Preprocessing Pipeline Development}: ...; \textbf{Preprocessing Pipeline Testing}: ...; \textbf{Data Augmentation Pipeline Testing}: ...; \textbf{Neural Network Building}: ...; \textbf{Neural Network Testing}: ...; \textbf{Drafting \& Editing}: Ilia Golub, Prosperity Oguama. All authors contributed to the study, read, revised and approved the final manuscript.

\subsection*{Data availability statement}
The data underlying this article is publicly available and accessible through \hyperlink{Schizconnect}{http://schizconnect.org/}.

\subsection*{Financial disclosure}

The authors report no financial disclosure.

\subsection*{Conflict of interest}

The authors declare no potential conflict of interests.

\end{document}